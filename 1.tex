\documentclass[../main.tex]{subfiles}


\begin{document}
    \begin{itemize}
        \item \textbf{LAN} - Local Area Network
        \item \textbf{MAN} - Metropolitan Area Network
        \item \textbf{WAN} - Wide Area Network
    \end{itemize}


    \subsection{Topologia sieci lokalnych}
    \begin{itemize}
        \item hierarchiczna
        \item \textit{(rozszerzonej)} gwiazdy
        \item pierścienia
        \item magistrali
    \end{itemize}
    Dwa rodzaje topologii: \textbf{fizyczne i logiczne}.
    Jeżeli przy fizycznej topologii gwiazdy komputer przesyła dane bezpośrednio
    do komputera docelowego (przełącznik), to mamy logiczną topologię gwiazdy.
    Jeżeli ramka jest wysyłana do wszystkich dostępnych komputerów
    (koncentrator), to logicznie jest to topologia magistrali.

    \textbf{Komutacja obwodów} - technologia sidn, „wersja cyfrowa telefonii“,
    jak się dzwoniło to się robiło logiczno/fizyczne stałe połączenie trwające
    tak długo jak trwa rozmowa

    \textbf{Komutacja  pakietów} - podzielone dane na pakiety, pakiety wysyłane
    różnymi ścieżkami

    %Ogólnie więcej: http://mml.pl/TT/2-komutacja/
    %lub http://www.scritub.com/limba/poloneza/Techniki-komutacji114127181.php

\end{document}
