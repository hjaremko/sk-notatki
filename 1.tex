\documentclass[../main.tex]{subfiles}


\begin{document}
    \begin{itemize}
        \item LAN - Local Area Network
        \item MAN - Metropolitan Area Network
        \item WAN - Wide Area Network
    \end{itemize}


    \subsection{Topologia sieci lokalnych}
    \begin{itemize}
        \item hierarchiczna
        \item (rozszerzonej) gwiazdy
        \item pierścienia
        \item magistrali
    \end{itemize}


    Dwa rodzaje topologii:
    \begin{itemize}
        \item Topologie fizyczne
        \item Topologie logiczne
    \end{itemize}

    Jeżeli przy fizycznej topologii gwiazdy komputer przesyła dane bezpośrednio do komputera docelowego (przełącznik), to mamy logiczną topologię gwiazdy. Jeżeli ramka jest wysyłana do wszystkich dostępnych komputerów (koncentrator), to logicznie jest to topologia magistrali.


    Komutacja obwodów - technologia sidn, „wersja cyfrowa telefonii“, jak się dzwoniło to się robiło logiczno/fizyczne stałe połączenie trwające tak długo jak trwa rozmowa
    Komutacja  pakietów - podzielone dane na pakiety, pakiety wysyłane różnymi ścieżkami



\end{document}