\documentclass[../main.tex]{subfiles}


\begin{document}
    Hierarchiczny rozproszony system przechowującym informacje o \textbf{nazwach komputerów} i
    ich numerach IP, który odpowiada na zapytania o nazwy domen.

    Za koordynację nazw domen i przypisywanie adresów IP jest \textbf{IANA} (Internet Assigned Numbers
    Authority) działająca pod ICANNem (Internet Corporation for Assigned Names and Numbers).

    \textbf{Domeny górnego poziomu}
    \begin{itemize}
        \item \textbf{Arpa} - specjalna, wykorzystywana do odwzorowania adresów IP w nazwy.
        \item \textbf{Domeny podstawowe}, np: com, edu, gov.
        \item \textbf{Domeny geograficzne}), np: pl, uk, de.
    \end{itemize}

    \textbf{Domeny drugiego poziomu} - np: edu.pl, com.pl, co.uk, ac.uk.

    Obszar, inaczej \textbf{strefa} jest częścią systemu DNS, która jest \textbf{oddzielnie administrowana}.

    \textbf{Poszukiwania w DNS}
    \begin{itemize}
        \item \textbf{Proste}, „do przodu”– klient zna nazwę domenową, a chce uzyskać numer IP.
        \item \textbf{Odwrotne} – klient zna adres IP i chce uzyskać nazwę domenową; wykorzystuje domenę arpa.in-addr.
    \end{itemize}

    \subsection{Typy serwerów DNS}
    W każdej strefie musi być uruchomiony podstawowy serwer DNS oraz pewna liczba
    serwerów drugoplanowych, zapewniających usługi w razie awarii serwera podstawowego.

    \begin{itemize}
        \item \textbf{Serwer podstawowy} pobiera dane z pliku konfiguracyjnego, natomiast serwery
        drugoplanowe uzyskują dane od serwera podstawowego na drodze tzw. transferu strefy.
        \item \textbf{Serwery drugoplanowe} odpytują serwer podstawowy o dane w sposób
        regularny.
        \item \textbf{Serwery podręczne} (lokalne), których zadaniem jest zapamiętanie na pewien czas w
        pamięci podręcznej danych uzyskanych od innych serwerów tak, aby kolejne zapytania
        klientów mogły być obsłużone lokalnie.
    \end{itemize}

    \textbf{Podział ze względu na sposób uzyskania odpowiedzi poszukiwania}
    \begin{itemize}
        \item \textbf{Przeszukiwanie rekurencyjne} – klient oczekuje od serwera żądanej informacji. W przypadku,
        gdy serwer nie przechowuje żądanej informacji, sam znajduje ją na drodze wymiany
        komunikatów z innymi serwerami.
        \item \textbf{Przeszukiwanie iteracyjne} – występuje między lokalnym serwerem DNS a innymi serwerami
        DNS. Jeśli odpytywany serwer nie zna szukanego adresu IP, odsyła pytającego do innych
        serwerów.
    \end{itemize}

    \textbf{Komunikacja klienta z serwerem DNS}\\
    Przy odwołaniu do nazwy domenowej system zwykle najpierw sprawdza, czy nie jest to
    nazwa hosta lokalnego, następnie sprawdza plik hosts - o ile istnieje. Jeśli nie znajdzie odpowiedniego wpisu, to
    wysyłane jest zapytanie do pierwszego serwera DNS (adres w pliku
    konfiguracyjnym).

    \textbf{Dynamiczny DNS (DDNS)}\\
    Chyba najważniejsze wpisy w DNS dotyczą serwisów, np. www. Standardowo DNS obsługuje
    odwzorowanie nazw do statycznych adresów IP, można skonfigurować dynamiczne przez usługodawców,
    którzy przypisują nazwę do swojego IP i pewnego numeru portu, następnie zapytanie
    przekierowują do komputera ze zmiennym IP z ewentualną zmianą portu. Na komputerze ze zmiennym IP należy
    zainstalować odpowiedni program (klient DDNS), który będzie powiadamiał serwer DDNS o
    zmianach adresu IP.

    \subsection{Rekordy zasobów}
    Każdy serwer DNS przechowuje \textbf{informacje o tej części obszaru nazw DNS, dla której jest
    autorytatywny}. Informacje
    zapisywane są w postaci tzw. rekordów zasobów. Są to np: określenie adresu IPv4/6 dla hosta; nazwę kanoniczną jako nazwę domeny,
    rekord wymiany poczty.
    Dla zwiększenia wydajności serwer DNS może przechowywać również rekordy zasobów
    domen z innej części drzewa domen.

    \subsection{DHCP - Dynamic Host Configuration Protocol}
    Serwer DHCP \textbf{przydziela adresy IP dynamicznie}.
    Przydział dynamiczny numerów IP umożliwia pracę (ale nie jednoczesną) wielu komputerów z
    przydzielonym jednym numerem IP.

    \textbf{Serwer DHCP może wykorzystywać różne sposoby przypisywania adresów}:
    \begin{itemize}
        \item \textbf{przydział statyczny} IP do danego komputera (ustawienie „ręczne”, danemu adresowi
        MAC jest przypisywany stale jeden na stałe wybrany IP),
        \item \textbf{automatyczny przydział statyczny} przy pierwszym starcie komputera i kontakcie z
        serwerem,
        \item \textbf{przydział dynamiczny}, w którym serwer \textbf{wynajmuje} adres IP na określony czas.
    \end{itemize}
    DHCP umożliwia budowanie systemów konfigurujących się automatycznie.
    Oprócz przydzielenia adresu IP serwer DHCP przesyła do komputera klienta również inne
    dane konfiguracyjne, np. adres sieci, maskę sieci, adres domyślnego rutera (bramy).


\end{document}