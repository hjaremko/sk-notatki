\documentclass[../main.tex]{subfiles}


\begin{document}

    \textbf{ICMP (Internet Control Message Protocol)}
    \begin{itemize}
        \item \textbf{raportowanie routingu},
        \item dostarczanie informacji o \textbf{błędach} podczas przesyłania ze źródła do komputera docelowego,
        \item dostarczanie funkcji sprawdzających możliwość komunikacji komputerów wykorzystaniem protokołu IP,
        \item pomoc w automatycznej konfiguracji hostów.
    \end{itemize}
    Komunikaty ICMP wysyłane są \textbf{w pakietach IP}. W efekcie w ramce znajduje się nagłówek IP, nagłówek ICMP oraz dane komunikatu ICMP.\\

    \textbf{Struktura} komunikatu ICMP
    \begin{itemize}
        \item \textbf{Typ} (1 oktet)
        \item \textbf{Kod} (1 oktet)
        \item \textbf{Suma kontrolna} (2 oktety)
        \item \textbf{Dane charakterystyczne dla typu} (różna długość)
    \end{itemize}

    \textbf{Typy komunikatów ICMP}\\
    % \begin{tabularx}{\textwidth}{|c| >{\centering\arraybackslash}X|}
    \begin{tabularx}{\textwidth}{|c| X |}
        \hline
        0 & Odpowiedź echa (echo reply)\\
        \hline
        3 & Miejsce docelowe nieosiągalne (destination unreachable)\\
        \hline
        4 & Tłumienie źródła (source quench)\\
        \hline
        5 & Przekierowanie (redirect)\\
        \hline
        8 & Żądanie echa (echo request)\\
        \hline
        9 & Ogłoszenie routera (router advertisement)\\
        \hline
        10 & Wybór routera (router selection)\\
        \hline
        11 & Przekroczenie czasu (time exceeded)\\
        \hline
        12 & Problem parametru (parameter problem)\\
        \hline
    \end{tabularx}

    \textbf{Żądanie i odpowiedź echa}\\
    Cel – wysłanie prostego komunikatu do węzła IP i odebranie echa tego komunikatu. Bardzo
    użyteczne przy usuwaniu problemów i naprawianiu sieci. Narzędzia takie jak ping oraz tracert i traceroute używają tych komunikatów ICMP do
    uzyskania informacji o dostępności węzła docelowego.\\

    \subsection{Komunikaty ICMP o przekierowaniu}
    Komunikaty ICMP o przekierowaniu pozwalają hostom TCP/IP na konfigurację tylko jednego
    routera – bramy domyślnej nawet w sytuacji, gdy w sieci lokalnej są dwa lub więcej
    routerów, które są odpowiedzialne za pewne miejsca docelowe.
    Hosty mogą zacząć pracę z jedną domyślną trasą i uczyć się topologii sieci (w szczególności
    informacji o routowaniu) poprzez otrzymywanie komunikatów ICMP.

    Komunikaty ICMP o przekierowaniu powinny być generowane przez routery, ale korzystać z
    nich mogą tylko hosty.
    Jeśli datagram IP zostanie celowo usuwanięty przez router, to może być wysłany odpowiedni komunikat protokołu ICMP do nadawcy.



\end{document}
