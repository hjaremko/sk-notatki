\documentclass[../main.tex]{subfiles}
\begin{document}

    \subsection{Szerokość pasma}
    W sieciach komputerowych termin ten oznacza na ogół liczbę bitów, które mogłyby być
    przesłane w danej technologii w ciągu sekundy. Może oznaczać również częstotliwość zegara
    taktującego, wykorzystywanego w danej technologii sieciowej.

    \subsection{Przepustowość}
    Przepustowość jest miarą ilości użytecznej informacji dostarczonej z sukcesem przez ścieżkę
    komunikacyjną. Przepustowość odnosi się do mierzonej efektywności systemu, np. łącze o
    szerokości pasma 100 Mb/s może osiągnąć przepustowość np. 70 Mb/s, ze względu na
    implementację, wykorzystane protokoły, szyfrowanie itd.\\

    W zależności od tego czy uwzględnia się narzut związany z technologią, narzut związany z protokołami komunikacyjnymi definiuje się różne odmiany przepustowości.
    \begin{itemize}
        \item \textbf{Przepustowość maksymalna}
        \begin{itemize}
            \item maximum theoretical throughput - największa możliwa do przesłania liczba bitów danych w jednostce czasu; do tych bitów NIE wlicza się narzutu warstwy pierwszej i drugiej,
            \item peak measured throughput -wartość mierzona w rzeczywistym systemie lub na symulatorze w krótkim odcinku czasu,
            \item maximum sustained throughput - średnia wartość dla długotrwałych obciążeń.
        \end{itemize}
        \item \textbf{Efektywna przepustowość} (Goodput, przepustowość warstwy aplikacji)\\
        \begin{itemize}
            \item Mierzy liczbę bitów danych efektywnie przesłanych w warstwie aplikacji w jednostce czasu.
            \item Iloraz liczby bitów dostarczonych w warstwie aplikacji do czasu mierzonego od wysłania pierwszego do dostarczenia ostatniego przesłanego bitu.
            \item Nie zalicza się narzutu związanego z technologią warstwy drugiej i pierwszej, protokołami z warstw wyższych, niezawodnością.
        \end{itemize}
    \end{itemize}

    \subsection{Pozostałe}
    \begin{itemize}
        \item \textbf{Opóźnienie} - w sieciach z przełączaniem pakietów oznacza czas, jaki
        mija od wysłania pakietu do odebrania go przez adresata. Może być mierzony w jedną stronę
        (one way) lub w dwie (round-trip delay time).
        \item \textbf{Zmienność opóźnienia} - miara krótkotrwałych zmian w opóźnieniu.
        Duża zmienność ma bardzo niekorzystny wpływ np. na jakość transmitowanego dźwięku.
        \item \textbf{Czas propagacji} - czas w jakim sygnał pokona pewną odległość w
        medium transmisyjnym.
        \item \textbf{Czas transmisji} - czas jaki zajmuje umieszczenie pakietu w
        medium transmisyjnym.
        \item \textbf{Opóźnienie przekazania} - czas jaki mija od odebrania pakietu
        przez urządzenie (przekazujące pakiet) do momentu, gdy pakiet może być wysłany.
        \item \textbf{Czas przetwarzania} - czas w jakim pakiet jest przetwarzany w
        urządzeniu, obejmuje np. wykonanie zmian w nagłówkach, wyznaczenie routera następnego
        skoku itd.
        \item \textbf{Niezawodność sieci}
        \item \textbf{Współczynnik gubienia pakietów}, Packet Loss Ratio (PLR) - iloraz liczby straconych
        pakietów do całkowitej liczby pakietów przesłanych w pewnym czasie.
    \end{itemize}



\end{document}