\documentclass[../main.tex]{subfiles}


\begin{document}

    \textbf{Protokoły routowania wektora odległości} - oparte na \textbf{algorytmie Bellmana Forda} obliczania najkrótszych ścieżek w grafie, "odległości od sieci".\\

    \textbf{System autonomiczny} - to zbiór prefiksów (adresów sieci IP) pod \textbf{wspólną administracyjną kontrolą}, w którym utrzymywany jest spójny schemat trasowania


    \subsection{Niekorzystne zjawiska związane z routowaniem wg protokołów wektora odległości}
    \begin{itemize}
        \item \textbf{Pętle routowania.}
        \item \textbf{Efekt odbijania.}
        \item \textbf{Zliczanie do nieskończoności.}
    \end{itemize}

    Taktyki rozwiązania:
    \begin{itemize}
        \item \textbf{Dzielony horyzont (split horizon)}
        \begin{itemize}
            \item \textbf{Do łącza nie zostanie przekazana informacja o trasach wiodących przez to łącze}.
            \item Dzielony horyzont nie zawsze, ale zazwyczaj likwiduje \textbf{pętle routowania}.
        \end{itemize}

        \item \textbf{Split horizon with poison reverse}\\
        Informacja odwrotna(reverse) jest przekazywana, jednak z metryką nieskończoną.

        \item \textbf{Natychmiastowe/wymuszane aktualizacje (triggered updates)}
        \begin{itemize}
            \item W przypadku zmiany metryki trasy \textbf{musi nastąpić rozgłoszenie bez względu na okres
            rozgłoszeń} charakterystyczny dla danego protokołu.
            \item Powoduje to \textbf{szybszą zbieżność} i częściowo zapobiega \textbf{pętlom routowania}.
        \end{itemize}

        \item \textbf{Zegary hold-down (hold-down timers)}
        \begin{itemize}
            \item Router po otrzymaniu od sąsiada informacji o dezaktualizacji trasy włącza \textbf{specjalny zegar} (hold-down timer).
            \item Jeśli przed upływem czasu progowego nastąpi:
            \begin{itemize}
                \item \textbf{aktualizacja od tego samego sąsiada} na trasę aktywną, to \textbf{trasa} jest zaznaczana jako \textbf{aktywna}.
                \item router dostanie \textbf{od innego routera informację} o trasie do rozważanego miejsca docelowego z \textbf{metryką mniejszą bądź równą} tej zdezaktualizowanej, wówczas następuje \textbf{wpis zgłoszonej trasy}.
                \item router dostanie \textbf{od innego routera informację} o trasie do rozważanego miejsca docelowego z \textbf{metryką większą} od tej zdezaktualizowanej, taka trasa \textbf{nie jest brana pod uwagę}.
            \end{itemize}
        \end{itemize}

        \item \textbf{Zatruwanie tras}\\
        Przekazywanie informacji o trasach niedostępnych, przyspiesza uzyskanie zbieżności.
    \end{itemize}

    \subsection{Protokół RIP}
    \begin{itemize}
        \item Metryką w RIP jest \textbf{liczba skoków} do celu. Metryka 16 oznacza nieskończoność.
        \item Wysyła \textbf{cały wektor} odległości.

        \item Można ustawić trasę domyślną (adres 0.0.0.0).

        \item \textbf{Autosumaryzacja, system klasowy} adresacji ip.

        \item \textbf{Cztery zegary, liczniki} (timers, counters):
        \begin{itemize}
            \item \textbf{Update timer} (30s.) – po przesłaniu wektora odległości
            zegar jest zerowany. Po osiągnięciu 30 s. wysyłany jest następny wektor.
            \item \textbf{Invalid timer} (180s.) – za każdym razem jak router dostaje uaktualnienie pewnej trasy zegar ten dla trasy jest zerowany. Po osiągnięciu wartości progowej trasa jest zaznaczana jako niepoprawna, ale pakiety jeszcze są kierowane tą trasą.
            \item \textbf{Hold-down timer} (180s.) – po przekroczeniu wartości progowej
            przez invalid timer trasa jest ustawiana w stan hold-down. Trasa jest ustawiana w stan holddown również gdy router dostanie informację o tym, że sieć jest nieosiągalna (i nie ma innej, osiągalnej trasy).
            \item \textbf{Flush-timer} (240s.) – zegar dla trasy jest zerowany po otrzymaniu informacji o trasie. Po osiągnięciu czasu progowego trasa jest usuwana nawet, jeśli trasa jest jeszcze w stanie hold-down.
        \end{itemize}

        \item \textbf{Zalety RIP}
        \begin{itemize}
            \item \textbf{prostota} - procesor nie jest nadmiernie obciążony aktualizacją tablicy routowania i innymi działaniami,
            \item \textbf{łatwość} konfiguracji.
        \end{itemize}

        \item \textbf{Wady RIP}
        \begin{itemize}
            \item \textbf{wolne} rozprzestrzenianie się informacji o zmianach w topologii sieci (wolna zbieżność),
            \item stosunkowo częste (co 30s.) przesyłanie dużych porcji informacji w komunikatach RIP, co \textbf{obciąża} sieć.
            \item Wadą RIPv1 jest to, że nie daje możliwości przesyłania masek.
        \end{itemize}
    \end{itemize}

    \subsection{RIPv2}
    \begin{itemize}
        \item RIPv2 przekazuje maski podsieci, można stosować sieci \textbf{bezklasowe} i podsieci o zmiennym rozmiarze.
        \item Umożliwia prostą \textbf{autentykację} (przez hasła).
        \item Przekazuje adresy następnego skoku w komunikatach.
        \item Część wad RIP pozostała: 16 jako metryka oznaczająca nieskończoność, brak alternatywnych tras.
        \item Rozgłaszanie przez \textbf{multicast}.
    \end{itemize}

    \subsection{Protokół IGRP}
    \begin{itemize}
        \item \textbf{numer AS} taki sam
        we wszystkich routerach na danym obszarze z komunikacją IGRP
        \item obsługuje \textbf{adresy klasowe},
        \item metryka 24-bitowa w IGRP jest tworzona na podstawie wartości metryk cząstkowych oraz zmiennych
        określających wagę każdej użytej metryki.
        \begin{itemize}
            \item \textbf{Szerokość pasma} (bandwidth); oznacza liczbę bitów, jakie może transmitować w
            jednostce czasu dana technologia.
            \item \textbf{Opóźnienie} (delay) – czas wędrówki pakietu od źródła do celu.
            \item \textbf{Obciążenie} (load).
            \item \textbf{Niezawodność} (reliability); wartość liczona jest jako swoisty
            „procent” pakietów, które dotarły do następnego routera.
        \end{itemize}
        $metric = (K_1 * bandwidth + \frac{(K_2*bandwidth)}{(256 - load)} + K_3 * delay) *
        \frac{K_5}{(reliability + K_4)}$
        Standardowo $K_1 == K_3 == 1, K_2 == K_4 == K_5 == 0$.
        \item przesyłane są również wartości \textbf{MTU} – najmniejsze MTU na trasie do sieci oraz liczba skoków
        \item przechowywanych jest \textbf{kilka optymalnych tras} do pewnego miejsca docelowego, mogą być przechowywane informacje o trasach nieoptymalnych
        \item \textbf{NIE ma trasy domyślnej 0.0.0.0}. Są trasy zewnętrzne, przez tzw. '\textbf{router of last resort}'
        wybierane jeśli nie znaleziono żadnej innej.
    \end{itemize}

    \subsection{Protokół EIGRP}
    \begin{itemize}
        \item obsługuje \textbf{adresowanie bezklasowe},
        \item \textbf{numer AS}, taki sam w komunikujących
        się routerach
        \item \textbf{IGRP i EIGRP mogą ze sobą współpracować}, jeśli mają ten sam numer; nastąpi przeliczenie metryki;
        trasa z IGRP jest traktowana jak trasa zewnętrzna
        \item metryka 32 bitowa; aktualizacje zawierają liczbę skoków dla
        trasy, jednak liczba skoków nie jest brana pod uwagę przy wyliczaniu metryki
    \end{itemize}


    \textbf{Kluczowe technologie i idee wykorzystane w EIGRP}
    \begin{itemize}
        \item Wykrywanie \textbf{sąsiadów}.
        \item Diffusing Update Algorithm \textbf{DUAL}.
        \item \textbf{Tigger updates} po wykryciu zmiany lub nowego sąsiada.
        \item Komunikaty \textbf{HELLO}.
        \item Wyznaczają \textbf{succesorów, feasible succesorów}.
    \end{itemize}

    \textbf{Wybrane zalety EIGRP}
    \begin{itemize}
        \item \textbf{Minimalne zużycie szerokości pasma} gdy sieć jest stabilna. W czasie normalnego
        stabilnego działania sieci jedynymi wymienianymi pakietami pomiędzy węzłami EIGRP są
        pakiety HELLO.
        \item Wydajne wykorzystanie szerokości pasma w czasie uzyskiwania zbieżności. Po zmianie
        propagowane są jedynie zmiany, nie całe wektory odległości. Po wykryciu sąsiada
        uaktualnienie wysyłane jest tylko do niego (unicast).
        \item \textbf{Szybka zbieżność} po wykryciu zmiany w sieci.
    \end{itemize}


    EIGRP wykorzystuje specjalny \textbf{niezawodny protokół} w warstwie transportu – \textbf{Reliable
    Transport Protocol}.
    \begin{itemize}
        \item Aktualizacje są przesyłane niezawodnie na adres grupowy 224.0.0.10. Potwierdzenia są przesyłane na adres
        jednostkowy (unicast). Jeśli potwierdzenie z określonym numerem sekwencji nie nadejdzie w
        czasie RTO (Retransmission TimeOut), pakiet z aktualizacją jest retransmitowany, tym razem
        na adres jednostkowy.
        \item Zwykłe pakiety HELLO oraz potwierdzenia nie są potwierdzane.
        \item DUAL jest używany do wyznaczenia sukcesorów i wykonalnych
        sukcesorów określających \textbf{trasy zapasowe}.
        \item Mechanizm wyznaczania tras zapasowych zapewnia, że \textbf{nie ma w nich pętli routowania}.
    \end{itemize}


\end{document}