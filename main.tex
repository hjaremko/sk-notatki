% !TEX options=--shell-escape
\documentclass{unibooklet}

\usepackage{parskip}
\usepackage{tikz}
\usepackage{amsmath}
\usepackage[hidelinks]{hyperref}
\usepackage{subfiles}
\usepackage{graphicx}
\usepackage{float}
\usepackage{tabularx}

\title{%
\vskip 2cm
Sieci Komputerowe}
\author{Małgorzata Dymek}
\date{2018/2019}

\begin{document}
    \makeTitlePage
    \tableofcontents
    \pagebreak

    % \section{Laboratorium 1}
    % Zasady zaliczenia: trzy kolokwia (dwa pisane (adresacja + routing i switching)  + jedno praktyczne ( naprawienie zepsutych rzeczy ) )
    % \begin{itemize}
    %     \item dwa kolokwia pisemne (20 + 20)
    %     \begin{itemize}
    %         \item adresacja
    %         \item routing i switching
    %     \end{itemize}
    %     \item kolokwium praktyczne (30)
    %     \begin{itemize}
    %         \item naprawienie zepsutych rzeczy
    %     \end{itemize}
    %     \item obecność - 15*2 = 30
    % \end{itemize}
    % W sumie 100 punktów.
    %
    % \begin{tabular}{|c|c|}
    % \hline
    % 0 - 50 & ndst\\
    % \hline
    % 51 - 60 & dst\\
    % \hline
    % 61 - 70 & dst+\\
    % \hline
    % 71 - 80 & db\\
    % \hline
    % 81 - 90 & db+\\
    % \hline
    % 91 - 100 & bdb\\
    % \hline
    % \end{tabular}
    %
    %
    % Bridge - fizyczne połączenie do fizycznej karty sieciowej (kabel rj5) jakby jedna szyna (wszystko do fizycznej wpada do wirtualnej)
    % Host only - tylko z komputera bez wyjścia na zewnątrz
    % NAP - adres z fizycznego komputera, nie z wirtualki
    %
    % settings - vnet3 - nr wirtualnego switcha
    % brama domyślna - wyjście na świat - pc2
    %
    % \section{Laboratorium 2}
    % Routing
    %
    % menedżer urządzeń -> nr COMu -> PuTTy serial COM -> konsola - wpisać no ->
    %
    %
    % \begin{itemize}
    %     \item Z listwy SERIAL (numer monitora nad) kablem prostym do konsoli routera.
    %     \item Z listwy LAN (numer monitora nad) kablem prostym do f0/1
    %     \item Z f0/0 kablem prostym do switcha.
    %     \item Odpowiednio seriale zgodnie z rysunkiem tymi płaskimi niebieskimi kablami.
    %     \item Menedżer urządzeń - sprawdzamy numer COM.
    %     \item PuTTy - wpisujemy odpowiedni numer COM.
    %
    %     \item Cisco
    %     enable\\
    %     configure terminal\\
    %
    %     interface x/x\\
    %     ip address \textit{IP_addr maska}\\
    %     interafce loopback nr \\
    %     clock rate – tylko na serialach, nie trzeba ustawiać na Cisco\\
    %     encapsulation ppp – ustawienie enkapsulacji (tylko na serialach)\\
    %     encapsulation hdlc - ustawienie enkapsulacji (tylko na serialach)\\
    %     no shutdown\\
    % \end{itemize}


    \section{Charakterystyka sieci LAN, WAN. Topologie połączeń. Komutacja obwodów vs. komutacja
    pakietów i komutacja komórek.}
    \subfile{1}

    \section{Model ISO OSI, model TCP/IP.}
    \subfile{2}

    \section{Standaryzacja w sieciach komputerowych, co to są dokumenty RFC.}
    \subfile{3}

    \section{Ethernet: sposób dostępu do nośnika, ramki.}
    \subfile{4}

    \section{Ethernet: działanie przełączników i koncentratorów (podstawy).}
    \subfile{5}

    \section{Protokół IPv4: adresacja, pola w nagłówku, fragmentacja.}
    \subfile{6}

    \section{Multiemisja (multicast) w IPv4 (IGMP, IGMP-snooping, współpraca technologii Ethernet z
    multiemisją – adresy MAC multiemisji).}
    \subfile{7}

    \section{Protokół ARP.}
    \subfile{8}

    \section{Protokół ICMP.}
    \subfile{9}

    \section{Protokół UDP: charakterystyka, nagłówek.}
    \subfile{10}

    \section{Protokół TCP: charakterystyka, mechanizmy, nagłówek.}
    \subfile{11}

    \section{Protokoły routowania typu wektor odległości: sposób działania, wady i zalety,
    podstawowe parametry protokołów RIP, RIP2, IGRP, EIGRP. (M.in. pętle routowania,
    zliczanie do nieskończoności, dzielony horyzont, zegary).}
    \subfile{12}

    \section{Protokoły routowania stanu łącza: sposób działania, charakterystyka protokołu OSPF,
    rodzaje obszarów.}
    \subfile{13}

    \section{DNS.}
    \subfile{14}

    \section{Działanie przełączników Ethernet: tryby działania, protokół STP, sieci VLAN, łącza
    trunkingowe, przełączniki warstwy 3.}
    \subfile{15}

    \section{Podstawy kryptografii: szyfrowanie z kluczem symetrycznym, szyfrowanie z kluczem
    publicznym i prywatnym, funkcje skrótu, podpis cyfrowy, certyfikaty.}
    \subfile{16}

    \section{Bezpieczne protokoły: SSL, TLS, IPSec (ze szczególnym naciskiem na protokół IPSec).}
    \subfile{17}

    \section{Protokół IPv6: adresacja, nagłówki, mechanizmy, ICMPv6 (m.in. jak odnaleźć adres MAC
    na podstawie adresu IPv6), mechanizmy przejścia między IPv4 i IPv6, mobilny IP.}
    \subfile{18}

    \section{Charakterystyka protokołu BGP (w zakresie omówionym na wykładzie).}
    \subfile{19}

    \section{Podstawy programowania w interfejsie gniazd (w zakresie omówionym na wykładzie).}
    \subfile{20}

\end{document}
