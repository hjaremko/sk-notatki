\documentclass[../main.tex]{subfiles}


\begin{document}
    OSI (Open Systems Interconnection) utworzony przez Międzynarodową Organizację
    Normalizacyjną stanowi \textbf{model referencyjny}.


    \begin{tabularx}{\textwidth}{ |c|c| >{\centering\arraybackslash}X | }
        \hline
        Nr warstwy OSI & Nazwa warstwy OSI & Nazwa warstwy TCP/IP\\
        \hline
        7 & Aplikacji & Aplikacji\\
        6 & Prezentacji &\\
        5 & Sesji & \\
        \hline
        4 & Transportu & Transportu \\
        \hline
        3 & Sieci & Intersieci \\
        \hline
        2 & Łącza danych & Interfejsu sieciowego\\

        1 & Fizyczna & \\
        \hline
    \end{tabularx}

    \begin{itemize}
        \item \textbf{Warstwa fizyczna} - standard połączenia fizycznego, charakterystyki wydajnościowe nośników. Same media transmisyjne pozostają poza dziedziną jej
        zainteresowania (czasem określane są terminem warstwa zerowa).
        \item \textbf{Warstwa łącza danych} – grupowanie danych wejściowych (z warstwy fizycznej) w bloki zwane \textbf{ramkami} danych („jednostki
        danych usług warstwy fizycznej”), mechanizmy kontroli poprawności
        transmisji (FCS).
        \item \textbf{Warstwa sieci} - określenie trasy przesyłania
        danych między komputerami poza lokalnym segmentem sieci LAN, protokoły trasowane takie jak IP (ze stosu protokołów TCP/IP).
        \item \textbf{Warstwa transportu} - kontrola błędów i przepływu danych
        poza lokalnymi segmentami LAN, protokoły zapewniające
        komunikację procesów uruchomionych na odległych komputerach, protokoły TCP, UDP.
        \item \textbf{Warstwa sesji} - zarządzanie przebiegiem komunikacji podczas
        połączenia między komputerami.
        \item \textbf{Warstwa prezentacji} - kompresja, kodowanie i
        translacja między niezgodnymi schematami kodowania oraz szyfrowanie.
        \item \textbf{Warstwa aplikacji} - interfejs między aplikacjami a
        usługami sieci.
    \end{itemize}

    \subsection{Zestaw (stos) protokołów TCP/IP}

    \textbf{Protokoły z zestawu TCP/IP}
    \begin{itemize}
        \item warstwa aplikacji:  \textbf{TELNET, FTP, DNS}
        \item warstwa transportu: \textbf{TCP, UDP}
        \item warstwa internetowa: \textbf{IP - IPv4 i IPv6}
        \item \textbf{ARP} - tłumaczy adresy między warstwą internetową a warstwą interfejsu
        sieciowego, czasami zaliczany do tej ostatniej,
        \item \textbf{ICMP} - m.in. komunikaty o problemach,
        \item \textbf{IGMP} - komunikacja grupowa.

    \end{itemize}

    Dane przechodząc w dół stosu protokołów TCP/IP są opakowywane i otrzymują
    odpowiedni nagłówek. Porcje danych przesyłane w dół stosu mają różne
    nazwy:
    \begin{itemize}
        \item \textbf{Komunikat} - porcja danych utworzona w warstwie aplikacji i przesłana do warstwy transportu.
        \item \textbf{Segment} - porcja danych utworzona przez oprogramowanie implementujące protokół TCP w warstwie transportu. Zawiera w sobie komunikat.
        \item \textbf{Datagram UDP} - porcja danych utworzona przez oprogramowanie implementujące protokół UDP w warstwie transportu.
        \item \textbf{Datagram} - również porcja danych utworzona w warstwie internetowej przez oprogramowanie implementujące protokół IP. Datagram IP zawiera w sobie segment, bywa nazywany pakietem.
        \item \textbf{Ramka} - porcja danych utworzona na poziomie dostępu do sieci.
    \end{itemize}

    Sekwencja zdarzeń przy wysłaniu danych:
    \begin{itemize}
        \item Aplikacja przesyła dane do warstwy transportu.
        \item Dalszy dostęp do sieci realizowany jest przez TCP albo UDP.
        \item Segment lub datagram UDP przesyłany jest do warstwy IP, gdzie protokół IP dołącza między innymi informacje o adresach IP źródła i celu tworząc datagram IP (pakiet).
        \item Datagram z IP przechodzi do warstwy interfejsu sieciowego, gdzie tworzone są ramki. W sieci LAN ramki zawierają adres fizyczny (przypisany do karty sieciowej) otrzymany zprotokołu ARP.
        \item Ramka przekształcana jest w ciąg sygnałów, który zostaje przesłany przez sieć.
    \end{itemize}




\end{document}
