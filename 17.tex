\documentclass[../main.tex]{subfiles}


\begin{document}
    Bezpieczne protokoły mogą być wykorzystywane:
    \begin{itemize}
        \item \textbf{w warstwie aplikacji}- szyfrowanie komunikatów HTTPS, protokoły SSL, TLS,
        \item \textbf{między warstwą sieci a transportu} - szyfrowanie pakietów IP – protokół IPSec,
        \item \textbf{w warstwie łącza danych} - szyfrowanie ramek, np. WEP, WPA, WPA2 w sieciach bezprzewodowych.
    \end{itemize}


    \subsection{Protokół IPSec}
    \begin{itemize}
        \item \textbf{warstwa IP}
        \item może szyfrować dane pochodzące z dowolnej
        aplikacji, proces szyfrowania i deszyfrowania jest \textbf{niewidoczny} dla użytkownika
        \item \textbf{Authentication Headers} (AH) - sprawdzenie \textbf{autentyczności i integralności} danych.
        \item \textbf{Encapsulating Security Payloads} (ESP) - \textbf{szyfrowanie} danych, oraz autentyczność i integralność danych. ESP może być
        używany samodzielnie lub z AH.
    \end{itemize}

    Przed przesyłaniem danych strony komunikujące się uzgadniają szczegóły takie jak sposób
    uwierzytelniania, wymiana kluczy, algorytmy szyfrowania.

    Tryby działania IPSec (zarówno AH jak i ESP):
    \begin{itemize}
        \item \textbf{Tryb transportu} (w sieci lokalnej) między dwoma punktami końcowymi transmisji.
        \item \textbf{Tryb tunelowania} – szyfrowanie w niezabezpieczonej części sieci (np. dane między
        biurami przesyłane przez Internet).
    \end{itemize}

    \textbf{Metody uwierzytelniania} w IPSec:
    \begin{itemize}
        \item \textbf{Kerberos}
        \item \textbf{Oparty o certyfikaty cyfrowe}
        \item \textbf{Klucz dzielony} - przechowywany we właściwościach napis jednakowy dla obu
        komunikujących się stron.
    \end{itemize}

    \textbf{Polityki stosowania IPSec}
    \begin{itemize}
        \item \textbf{Client} (respond only) - transmisje bez IPSec, chyba że druga strona zażąda IPSec
        \item \textbf{Server} (request security) - żądanie transmisji IPSec, ale jeśli druga strona nie implementuje IPSec, to komunikacja bez IPSec
        \item \textbf{Secure server} (require security) - żądanie transmisji IPSec, jeśli druga strona nie implementuje IPSec, to komunikacja nie jest kontynuowana.
    \end{itemize}

    \textbf{Filtry IPSec}\\
    Filtr IPSec pozwala na automatyczne przepuszczenie datagramów IP, blokowanie lub użycie
    negocjacji (i w konsekwencji użycie IPSec) w zależności od źródła i miejsca docelowego IP,
    protokołu transportowego, portów źródłowych i docelowych.


    \subsection{SSL - Secure Socket Layer}
    \begin{itemize}
        \item \textbf{warstwa aplikacji} TCP/IP (prezentacji w ISO/OSI)
        \item jego zadaniem jest \textbf{zabezpieczanie informacji} przesyłanych siecią.
        \item zapewnia autoryzację serwerów internetowych i (opcjonalnie) klientów (utrudnia
        podszywanie pod autoryzowanych usługodawców i użytkowników)
        \item często prezentowany jako protokół, który leży
        powyżej warstwy transportu (TCP, UDP) i sieci (IP) a poniżej warstwy aplikacji (np. HTTP, FTP,
        SMTP, TELNET)
        \item jest protokołem \textbf{otwartym}
        \item wykorzystuje \textbf{szyfrowanie symetryczne} z kluczem \textbf{publicznym}
        \item protokoły zabezpieczone SSL oznaczane są jako HTTPS (dla HTTP), FTPS (dla FTP) itd.
        \item stosuje \textbf{sumy kontrolne} dla zapewnienia \textbf{integralności}.
    \end{itemize}

    Po nawiązaniu połączenia następuje wymiana informacji (certyfikatów CA i kluczy publicznych)
    uwierzytelniających serwera i (opcjonalnie) klienta.
    Serwer i klient uzgadniają również algorytmy szyfrowania – najsilniejsze dostępne jednocześnie obu stronom.
    Następnie serwer i klient generują klucze sesji (symetryczne), które są szyfrowane kluczem
    publicznym drugiej strony. Klucze sesji są odszyfrowywane przy pomocy klucza prywatnego i
    następnie służą do szyfrowania danych.

    \subsection{TLS - Transport Layer Security}
    \begin{itemize}
        \item \textbf{warstwa aplikacji}
        \item strony dogadują się co do klucza symetrycznego szyfrowaniem niesymetrycznym
        \item symetryczne szyfrowanie danych
    \end{itemize}

\end{document}