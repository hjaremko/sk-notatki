\documentclass[../main.tex]{subfiles}


\begin{document}


    \subsection{TCP – Transmission Control Protocol}
    Ilość bajtów danych przesyłanych w jednym segmencie nie powinna być większa niż ustalony MSS (\textbf{Maximum Segment Size}).\\

    \textbf{Cechy TCP}
    \begin{itemize}
        \item Partnerzy (procesy) tworzą połączenie z wykorzystaniem mechanizmu (trójfazowego) uzgodnienia.
        \item Zamknięcie połączenia odbywa się z wykorzystaniem mechanizmu uzgodnienia (zgoda na zamknięcie).
        \item TCP zapewnia sterowanie przepływem. Informuje partnera o tym ile bajtów danych ze strumienia danych może od niego przyjąć (okno oferowane). Rozmiar okna zmienia się dynamicznie i jest równy rozmiarowi wolnego miejsca w buforze odbiorcy. Zero oznacza, że nadawca musi zaczekać, aż program użytkowy
        odbierze dane z bufora.
        \item Dane ze strumienia danych dzielone są na fragmenty, które według TCP mają najlepszy
        do przesłania rozmiar. Jednostka przesyłania danych nazywa się \textbf{segmentem}.
        \item TCP zapewnia \textbf{niezawodność} połączenia.
    \end{itemize}

    \textbf{Mechanizmy niezawodności}
    \begin{itemize}
        \item \textbf{Potwierdzanie otrzymania segmentów z mechanizmem zegara.}\\
        Odebrany segment musi być potwierdzony przez odbiorcę przez wysłanie segmentu potwierdzającego. Jeśli
        potwierdzenie nie nadejdzie w odpowiednim czasie, segment zostanie przesłany powtórnie.
        \item \textbf{Sumy kontrolne.}\\
        Jeśli segment zostanie nadesłany z niepoprawna sumą kontrolną, to jest
        odrzucany. Nadawca po odczekaniu odpowiedniego czasu prześle segment jeszcze raz.
        \item \textbf{Przywracanie kolejności nadchodzących segmentów.}\\
        Segmenty mogą nadchodzić w kolejności innej niż zostały wysłane, oprogramowanie TCP przywraca prawidłową kolejność przed przekazaniem do aplikacji.
        \item \textbf{Odrzucanie zdublowanych danych.}
    \end{itemize}

    \subsubsection{Nagłówek TCP}
    \begin{itemize}
        \item \textbf{Numer sekwencji.}\\
        \item \textbf{Długość nagłówka} (przesunięcie danych).\\
        \item \textbf{Jednobitowe znaczniki} (flagi):
        \item \textbf{Rozmiar okna} - liczba bajtów, które odbiorca może zaakceptować.
        \item \textbf{Suma kontrolna.}\\
        \item \textbf{Wskaźnik ważności.}\\
        \item \textbf{Opcje} - rodzaj opcji (bajt), długość opcji (bajt), opcja. Najważniejsza opcja to \textbf{MSS}. Może być uzyskana jako MTU (Maximum Transmission Unit) minus rozmiar nagłówka IP oraz TCP.
    \end{itemize}

    \textbf{Specyfika stanu TIME WAIT}\\
    Spóźnione segmenty są w czasie 2 MSL odrzucane. Para
    punktów końcowych definiujących połączenie nie może być powtórnie użyta przed upływem
    2MSL. Eliminuje to ewentualne kłopoty związane z odbieraniem z sieci segmentów jeszcze ze
    starego połączenia.

    \textbf{Półzamknięcie TCP}\\
    Strona, która zakończyła połączenie i nie nadaje danych, może dane odbierać od partnera
    TCP. Takie połączenie nazywane jest połączeniem półzamkniętym (half-closed).

    \textbf{Segmenty RST}\\
    Segment RST wysyłany jest przez oprogramowanie implementujące TCP, kiedy nadchodzi
    segment niepoprawny z punktu widzenia dowolnego połączenia. Segment RST nie jest potwierdzany. W protokole UDP generowany jest komunikat ICMP o tym, że port jest nieosiągalny.
    Segment RST jest wysyłany również wtedy, gdy przekroczona jest maksymalna dopuszczalna
    liczba połączeń TCP.


    \textbf{Połączenia półotwarte} (połowicznie otwarte)\\
    Jest to połączenie nie poprawnie nawiązane. Występuje, jeśli jedna ze stron przerwała połączenie bez informowania drugiej. Segment z ustawioną na 1 flagą SYN został przesłany od
    klienta do serwera, serwer odpowiedział segmentem z ustawionymi na 1 flagami SYN i ACK,
    ale klient nie odpowiedział segmentem z ustawioną na 1 flagą ACK.
    Jeden ze sposobów atakowania serwisów (np. WWW) polegał na otwieraniu bardzo dużej
    liczby połączeń półotwartych. Obecnie implementacje TCP są odporne na tego typu ataki.
    Dopuszczalne jest, by oprogramowanie realizujące TCP mogło sprawdzać stan połączenia
    przez okresowe przesyłanie segmentów sprawdzających aktywność. Segment taki to zawiera
    ustawioną na 1 flagę ACK i nie zawiera żadnych danych. Dodatkowo ma on ustawiony numer
    sekwencyjny na o 1 mniejszy od tego, którego normalnie spodziewa się strona wysyłająca
    ACK. Partner odpowiada też segmentem z ustawioną na 1 flagą ACK ze standardowo
    ustawionymi prawidłowymi wartościami numerów sekwencyjnych.



    \subsubsection{Przepływ danych w TCP}
    \textbf{Potwierdzenia}
    \begin{itemize}
        \item \textbf{Skumulowane potwierdzanie} - wysyłamy dużo segmentów, oczekujemy jednego skumulowanego potwierdzenia.
        \item \textbf{Opóźnione potwierdzenia} - serwer może wysłać potwierdzenie z opóźnieniem.
        \item \textbf{Selektywne potwierdzenia} - selektywnie potwierdzamy co dostaliśmy [przedziały], więc jeśli zginęło tylko kilka datagramów, to można retransmitować tylko je a nie całość.
    \end{itemize}

    \textbf{Ruchome okna TCP} (sliding windows)\\
    Połączenie TCP obejmuje dwa strumienie danych. W każdym strumieniu określony jest
    nadawca i odbiorca. Kontrolę przesyłania oktetów w strumieniu umożliwiają mechanizmy
    tzw. przesuwnych (ruchomych) okien, które można sobie wyobrazić jako nałożone na
    strumień. Dla strumienia określone jest okno nadawcy oraz okno odbiorcy. Nadawca może
    wysyłać tylko te dane, które są w tej chwili w jego oknie nadawczym, przy czym może to
    zrobić tylko za zgodą odbiorcy. Okno nadawcze jest przesuwane nad wyjściowym
    strumieniem bajtów, okno odbiorcze nad strumieniem wejściowym.\\

    \subsubsection{Przesyłanie małych segmentów}
    Tak określa się segmenty o rozmiarze mniejszym od MSS.
    \begin{itemize}
        \item \textbf{Algorytm Nagle’a}\\
        „Dopasowuje się” do sieci, w której przesyłane są segmenty.
        \begin{itemize}
            \item małe niepotwierdzone segmenty są gromadzone w buforze, wysyłane razem.
        \end{itemize}
        Algorytm Nagle’a może być wyłączany przez oprogramowanie TCP.
        \item \textbf{Syndrom głupiego okna} (SWS)\\
        \begin{itemize}
            \item Jeśli odbiorca ma zerowy rozmiar okna (i nadawca też) oraz warstwa aplikacji pobierze 1
            bajt, to okno odbiorcze otwiera się o jeden bajt.
            \item Nadawca unika SWS wstrzymując się z wysyłaniem danych dopóki rozmiar okna proponowanego przez odbiorcę nie jest równy co najmniej MSS.
        \end{itemize}
    \end{itemize}

    \textbf{Dodatkowa kontrola przepływu po stronie nadawcy}\\
    \begin{itemize}
        \item \textbf{Algorytm powolnego startu}\\
        Po otwarciu połączenia lub dłuższym czasie nie przesyłania danych wielkość okna
        przeciążeniowego ustawiana jest na 2*MSS. Każde przychodzące potwierdzenie (ACK)
        powoduje zwiększenie okna przeciążeniowego o jeden MSS. Może to prowadzić do wykładniczego wzrostu wielkości tego okna.
        \item \textbf{Algortym unikania zatoru}\\
        Tu stosuje się wolniejszy wzrost wielkości okna przeciążeniowego, np. o jeden
        segment na kilka przychodzących ACK. Algorytm ten działa zwykle od pewnego progu (najpierw działa powolny start).
    \end{itemize}


    \subsubsection{Retransmisje segmentów w TCP}
    W każdym połączeniu definiowana jest zmienna RTO (Retransmission Time-out). Jeśli TCP nie
    odbierze ACK w czasie RTO dla pewnego nadanego segmentu, to segment musi być
    retransmitowany.


\end{document}