\documentclass[../main.tex]{subfiles}


\begin{document}

        Oprócz adresu IP komputer ma
        przyporządkowaną \textbf{nazwę}. Konwencje nazywania komputerów:
        \begin{itemize}
            \item \textbf{nazwy hosta}
            \item \textbf{nazwy DNS}
            \item \textbf{nazwy NetBIOS}
        \end{itemize}

        Istnieją mechanizmy tłumaczące nazwy na numery IP i odwrotnie.
        Ciałem odpowiedzialnym za koordynację nazw domen górnego poziomu a także
        odpowiedzialnym za przypisywanie adresów IP jest IANA Internet Assigned Numbers
        Authority.
        Ciałem nadzorującym od strony technicznej różne działania związane z uzyskiwaniem
        (rejestrowaniem) nazw domen, numerów Ip, numerów portów jest ICANN - Internet
        Corporation for Assigned Names and Numbers.

        \textbf{Domeny górnego poziomu}
        \begin{itemize}
            \item arpa - specjalna, wykorzystywana do odwzorowania adresów IP w nazwy.
            \item Domeny podstawowe (generic, gTLD), np: com, edu, gov.
            \item Domeny geograficzne (krajowe, country-code ccTLD), np: pl, uk, de.
        \end{itemize}

        \textbf{Domeny drugiego poziomu} - w wielu krajach domeny drugiego poziomu odzwierciedlają
        domeny organizacyjne pierwszego poziomu, ale ujmowane na swoim terytorium. Przykłady:
        edu.pl, com.pl, co.uk, ac.uk.

        Obszar, inaczej \textbf{strefa} (zone) jest częścią systemu DNS, która jest oddzielnie administrowana.
        Domeny drugiego poziomu dzielone są na mniejsze strefy. Z kolei te strefy mogą być dalej
        dzielone. Występuje tu delegowanie zarządzania w dół struktury drzewa. Jednostka
        odpowiedzialna za zarządzanie daną strefą decyduje ile będzie serwerów DNS w strefie,
        rejestruje i udostępnia nazwy i numery IP nowych komputerów zainstalowanych w strefie.
        W tej chwili jest na świecie 13 (typów) serwerów głównych (najwyższego poziomu) zwanych
        po angielsku root-servers, posiadającymi nazwy od a.root-servers.net do m.root-servers.net.

        \textbf{Poszukiwania w DNS}
        \begin{itemize}
            \item Proste, „do przodu”– klient zna nazwę domenową, a chce uzyskać numer IP.
            \item Odwrotne (reverse) – klient zna adres IP i chce uzyskać nazwę domenową.
            Przeszukiwanie odwrotne wykorzystuje domenę arpa.in-addr. Jeśli chcemy
            poznać nazwę domenową komputera, to w systemie DNS adres ten
            jest reprezentowany jako specyficzna nazwa w domenie arpa.in-addr.
        \end{itemize}

        \subsection{Typy serwerów DNS}
        W każdej strefie musi być uruchomiony podstawowy serwer DNS oraz pewna liczba
        serwerów drugoplanowych, zapewniających usługi w razie awarii serwera podstawowego.
        Serwer podstawowy pobiera dane z pliku konfiguracyjnego, natomiast serwery
        drugoplanowe uzyskują dane od serwera podstawowego na drodze tzw. transferu strefy
        (zone transfer). Serwery drugoplanowe odpytują serwer podstawowy o dane w sposób
        regularny, zwykle co kilka godzin. Oprócz dwóch wymienionych rodzajów serwerów są
        jeszcze serwery podręczne (lokalne), których zadaniem jest zapamiętanie na pewien czas w
        pamięci podręcznej danych uzyskanych od innych serwerów tak, aby kolejne zapytania
        klientów mogły być obsłużone lokalnie.
        Serwery DNS działają na portach 53 UDP oraz 53 TCP. Na ogół w warstwie transportu
        używany jest UDP. Wyjątkiem jest m.in. transmisja danych z serwera podstawowego do
        drugoplanowego (większe porcje danych) oraz komunikaty w sieciach WAN. Również kiedy w
        odpowiedzi od serwera (przez UDP) ustawiony jest bit TC (patrz niżej) ponawiane jest
        zapytanie z wykorzystaniem TCP.

        \textbf{Podział ze względu na sposób uzyskania odpowiedzi poszukiwania}
        \begin{itemize}
            \item Przeszukiwanie rekurencyjne – klient oczekuje od serwera żądanej informacji. W przypadku,
            gdy serwer nie przechowuje żądanej informacji, sam znajduje ją na drodze wymiany
            komunikatów z innymi serwerami.
            \item  Przeszukiwanie iteracyjne – występuje między lokalnym serwerem DNS a innymi serwerami
            DNS. Jeśli odpytywany serwer nie zna szukanego adresu IP, odsyła pytającego do innych
            serwerów (odpowiedzialnych za daną domenę).
        \end{itemize}

        \textbf{Komunikacja klienta z serwerem DNS}\\
        Przy odwołaniu do nazwy domenowej system zwykle najpierw sprawdza, czy nie jest to
        nazwa hosta lokalnego, następnie sprawdza plik hosts - o ile istnieje. Jeśli nie znajdzie odpowiedniego wpisu, to
        wysyłane jest zapytanie do pierwszego serwera DNS (adres w pliku
        konfiguracyjnym).

        \textbf{Standardowy sposób poszukiwania}\\
        Klient pyta swój domyślny serwer DNS wysyłając zapytanie rekurencyjne.
        Odpytany serwer realizuje zapytania iteracyjne, zaczynając od serwerów głównych, które
        odsyłają do serwerów niższego poziomu.
        Mechanizm ten może się wydawać nieefektywny, ale w rzeczywistości dzięki temu, że
        serwery DNS zapamiętują na pewien czas informacje uzyskane z innych serwerów DNS
        (cache), często odpowiedź na zapytanie programu-klienta zostaje znaleziona bardzo szybko.

        \textbf{Dynamiczny DNS (DDNS)}\\
        Chyba najważniejsze wpisy w DNS dotyczą serwisów, np. www. Standardowo DNS obsługuje
        odwzorowanie nazw do statycznych adresów IP. Można jednak skonfigurować odwzorowanie dla adresów zmieniających się
        dynamicznie. W tym celu należy skorzystać z odpowiednich usługodawców w Internecie,
        którzy przypisują nazwę do swojego IP i pewnego numeru portu, następnie zapytanie
        przekierowują do komputera ze zmiennym IP z ewentualną zmianą portu. Na komputerze ze zmiennym IP należy
        zainstalować odpowiedni program (klient DDNS), który będzie powiadamiał serwer DDNS o
        zmianach adresu IP.
        Oddzielnym problemem, który należy rozwiązać, jest wykorzystanie serwera NAT i
        przypisywanie adresów prywatnych do serwisu w sieci. Na ogół wystarczy odpowiednie
        działanie klienta DDNS oraz odpowiednie skonfigurowanie serwera NAT.


        \subsection{Rekordy zasobów}
        Każdy serwer DNS przechowuje informacje o tej części obszaru nazw DNS, dla której jest
        autorytatywny (administratorzy są odpowiedzialni za poprawność informacji). Informacje
        zapisywane są w postaci tzw. rekordów zasobów.
        Dla zwiększenia wydajności serwer DNS może przechowywać również rekordy zasobów
        domen z innej części drzewa domen.
        \textbf{Istnieje szereg typów rekordów zasobów:}\\
        SOA (Start of Authority) Rekord uwierzytelnienia – pierwszy rekord w pliku
        strefy, określa podmiot odpowiedzialny od tego punktu hierarchii „w dół”.
        \begin{itemize}
            \item Serial – pole zawierające numer wersji pliku strefy, zwykle w polu tym odzwierciedlona jest
            data oraz numer wersji pliku w danym dniu.
            \item Refresh – określa jak często serwer pomocniczy ma sprawdzać na serwerze podstawowym,
            czy nie zachodzi potrzeba uaktualnienia plików.
            \item Retry – czas, po którym serwer pomocniczy będzie ponownie próbował odtworzyć dane po
            nieudanej próbie odświeżenia.
            \item Expire – maksymalny limit czasu, przez który serwer pomocniczy może utrzymywać dane w
            pamięci cache bez ich uaktualnienia.
            Minimum (Default TTL) – domyślny czas, jaki ma być użyty dla rekordów, które nie mają
            określonego TTL.
        \end{itemize}

        \subsection{DHCP - Dynamic Host Configuration Protocol}
        Wadą BOOTP jest statyczny sposób przydzielania numerów IP.
        Przydział dynamiczny umożliwia pracę (ale nie jednoczesną) wielu komputerów z
        przydzielonym jednym numerem IP.
        Serwer DHCP przydziela adresy IP dynamicznie. Obecnie w bardzo wielu sieciach lokalnych
        komputery nie mają na stałe wpisanych IP, ale pobierają IP od serwera DHCP w momencie
        startu systemu.
        Serwer DHCP może wykorzystywać różne sposoby przypisywania adresów:
        \begin{itemize}
            \item przydział statyczny IP do danego komputera (ustawienie „ręczne”, danemu adresowi
            MAC jest przypisywany stale jeden na stałe wybrany IP),
            \item automatyczny przydział statyczny przy pierwszym starcie komputera i kontakcie z
            serwerem,
            \item przydział dynamiczny, w którym serwer wynajmuje adres IP na określony czas.
            DHCP umożliwia budowanie systemów konfigurujących się automatycznie.
            Oprócz przydzielenia adresu IP serwer DHCP przesyła do komputera klienta również
        \end{itemize}


\end{document}