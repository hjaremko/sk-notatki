\documentclass[../main.tex]{subfiles}
\begin{document}


    Sposób obsługi routowania przez warstwę IP to \textbf{mechanizm routowania} -
    przeglądanie przez tablicy routowania, podejmowanie decyzji co do przesyłania datagramów IP.
    Przez pojęcie \textbf{polityka routowania} określa się działania procesu routowania podejmowane w
    celu ustanowienia i bieżącej modyfikacji tablicy routowania, jest ona realizowana z wykorzystaniem protokołów routowania.\\

    Pożądane cechy protokołów routowania to:
    \begin{itemize}
        \item Wyznaczenie \textbf{najlepszej trasy} do punktu docelowego, wymaga określenia kryterium porównywania tras.
        \item \textbf{Odporność} (robustness) - protokoły muszą zawsze działać poprawnie.
        \item \textbf{Szybkie osiągnięcie zbieżności} (rapid convergence), czyli stanu, w którym wszystkie routery „widzą” jednakowo topologię sieci.
        Szybkość określa czas rozpowszechnienia informacji o zmianach.
        \item \textbf{Dopasowanie do zmian} (flexibility), wyznaczanie nowych optymalnych tras.
    \end{itemize}

    Internet jest zorganizowany jako grupa tzw. systemów autonomicznych (Autonomous
    System – AS), z których każdy jest osobno administrowany. W każdym są wewnętrzne protokoły routowania (ang. IGP – interdomain gateway protocol lub IRP – interdomain routing protocol).\\

    Ze względu na sposób działania protokoły routowania wewnętrznego dzielimy na:
    \begin{itemize}
        \item protokoły \textbf{wektora odległości} (DV, distance-vector): RIP, RIP2, IGRP, EIGRP
        \item protokoły \textbf{stanu łącza} (link-state): OSPF, OSPF2, IS-IS, NLSP
    \end{itemize}
    Uwaga: w starszych opracowaniach firmy Cisco protokół EIGRP był określany jako protokół
    hybrydowy.

    \textbf{Zewnętrzne protokoły routowania} (EGP -bexterior gateway protocols) - międzydomenowe protokoły routowania używane między routerami działającymi w różnych systemach autonomicznych.
    Najważniejszym protokołem zewnętrznym jest BGP.
    Oprócz protokołów routowania rozważa się protokoły routowalne, takie jak IP, IPX, Apple
    Talk. Określają sposoby adresowania, umożliwiające dostarczanie
    pakietów w złożonej sieci komputerowej. Routery mogą wyznaczać trasy dla różnych
    protokołów routowalnych, nie tylko IP. Każdy z protokołów routowania i każdy z protokołów
    routowalnych musi być w routerze skonfigurowany.

    \subsection{Protokoły routowania wektora odległości}
    Distance vector – wektor odległości. Protokoły wektora odległości są oparte na \textbf{algorytmie Bellmana Forda} obliczania najkrótszych ścieżek w grafie.\\
    Węzły grafu oznaczają routery, krawędzie odpowiadają połączeniom między routerami.
    Połączenia te mają różne koszty, co odpowiada różnym wagom krawędzi grafu.
    W protokołach routowania wagi nie mogą być ujemne, co oznacza, że można też
    wykorzystać np. algorytm Dijkstry (jest wykorzystywany w protokołach stanu łącza).\\

    Nie zawsze router jest w stanie wykryć uszkodzenie łącza. Uszkodzeniu
    może ulec też np. sąsiedni router albo ramka zawierająca pakiet z wektorem odległości i wektor
    odległości nie dotrze do docelowego routera. Dlatego w protokołach typu wektor odległości
    trasa jest zaznaczana jako niedostępna, gdy router nie dostanie o niej informacji od sąsiada
    przez kilka kolejnych rozgłoszeń (np. w RIP 180 sekund, czyli sześć
    rozgłoszeń). Trasa niedostępna nie jest jeszcze usuwana z tablicy routowania przez
    kilka rozgłoszeń, np. w RIP przez 90 sekund. Powoduje to większe opóźnienie w czasie uzyskania zbieżności.\\

    \subsection{Niekorzystne zjawiska związane z routowaniem wg protokołów wektora odległości}
    \begin{itemize}
        \item Pętle routowania.
        \item Efekt odbijania.
        \item Zliczanie do nieskończoności.
    \end{itemize}

    Taktyki rozwiązania:
    \begin{itemize}
        \item \textbf{Dzielony horyzont (split horizon)}
        \begin{itemize}
            \item \textbf{Do łącza nie zostanie przekazana informacja o trasach wiodących przez to łącze}.
            \item Dzielony horyzont nie zawsze, ale zazwyczaj likwiduje \textbf{pętle routowania}.
        \end{itemize}

        \item \textbf{Natychmiastowe/wymuszane aktualizacje (triggered updates)}
        \begin{itemize}
            \item W przypadku zmiany metryki trasy \textbf{musi nastąpić rozgłoszenie bez względu na okres
            rozgłoszeń} charakterystyczny dla danego protokołu.
            \item Powoduje to \textbf{szybszą zbieżność} i częściowo zapobiega \textbf{pętlom routowania}.
        \end{itemize}

        \item \textbf{Zegary hold-down (hold-down timers)}
        \begin{itemize}
            \item Router po otrzymaniu od sąsiada informacji o dezaktualizacji trasy włącza \textbf{specjalny zegar} (hold-down timer).
            \item Jeśli przed upływem czasu progowego nastąpi:
            \begin{itemize}
                \item \textbf{aktualizacja od tego samego sąsiada} na trasę aktywną, to \textbf{trasa} jest zaznaczana jako \textbf{aktywna}.
                \item router dostanie \textbf{od innego routera informację} o trasie do rozważanego miejsca docelowego z \textbf{metryką mniejszą bądź równą} tej zdezaktualizowanej, wówczas następuje \textbf{wpis zgłoszonej trasy}.
                \item router dostanie \textbf{od innego routera informację} o trasie do rozważanego miejsca docelowego z \textbf{metryką większą} od tej zdezaktualizowanej, taka trasa \textbf{nie jest brana pod uwagę}.
            \end{itemize}
        \end{itemize}
    \end{itemize}

    \subsection{Protokół RIP}
    \begin{itemize}
        \item Protokół typu wektor odległości.

        \item Metryką w RIP jest liczba skoków (hops) do celu. Można skonfigurować inne wartości dla połączeń, więc np. preferencję tras szybszych.

        \item Metryka 16 oznacza umownie nieskończoność (miejsce niedostępne), zatem RIP nie jest dobrym protokołem w przypadku dużych sieci.

        \item Można ustawić trasę domyślną (adres 0.0.0.0).

        \item W wersji RIP na routerach Cisco można przechowywać więcej niż jedną trasę o
        takiej samej metryce. Można włączyć równoważenie obciążeń (load balancing) na dwa
        sposoby:
        \begin{itemize}
            \item \textbf{Process switching} (packet-by-packet load balancing), kosztowny i dlatego nie polecany, każdy pakiet jest kierowany osobno, dla każdego jest przeglądana tablica routowania.
            \item \textbf{Fast switching} (destination-by-destination), tylko dla pierwszego pakietu z pewnego miejsca źródłowego do pewnego miejsca docelowego przeszukiwana jest tablica routowania. Wyznaczona trasa jest zapisywana w pamięci podręcznej (cache) i kolejne pakiety wędrują tą samą ścieżką.
        \end{itemize}

        \item Wykorzystywane są następujące \textbf{mechanizmy} typowe dla protokołów wektor odległości:
        \begin{itemize}
            \item Split horizon (+ with poison reverse)
            \item Holddown counters (timers)
            \item Triggered updates
        \end{itemize}

        \item \textbf{Cztery zegary, liczniki} (timers, counters):
        \begin{itemize}
            \item \textbf{Update timer} (standardowo 30 sekund) – po przesłaniu wektora odległości
            (routing update) zegar jest zerowany. Po osiągnięciu 30 s. wysyłany jest następny wektor.
            \item \textbf{Invalid timer} (standardowo 180 sekund) – za każdym razem jak router dostaje uaktualnienie pewnej trasy zegar ten dla trasy jest zerowany. Po osiągnięciu wartości progowej trasa jest zaznaczana jako niepoprawna, ale pakiety jeszcze są kierowane tą trasą.
            \item \textbf{Hold-down timer} (standardowo 180s.) – po przekroczeniu wartości progowej
            przez invalid timer trasa jest ustawiana w stan hold-down. Trasa jest ustawiana w stan holddown również gdy router dostanie informację o tym, że sieć jest nieosiągalna (i nie ma innej, osiągalnej trasy).
            \item \textbf{Flush-timer} (standardowo 240s.) – zegar dla trasy jest zerowany po otrzymaniu informacji o trasie. Po osiągnięciu czasu progowego trasa jest usuwana nawet, jeśli trasa jest jeszcze w stanie hold-down.
        \end{itemize}

        \item Przewidziano możliwość \textbf{odpytywania routera o cały wektor odległości} lub o \textbf{trasy} do pewnych miejsc docelowych – takie możliwości są wykorzystywane prze starcie oraz na ogół w celach diagnostycznych.\\

        \item \textbf{Zalety RIP}
        \begin{itemize}
            \item prostota - procesor nie jest nadmiernie obciążony aktualizacją tablicy routowania i innymi działaniami,
            \item łatwość konfiguracji.
        \end{itemize}

        \item \textbf{Wady RIP}
        \begin{itemize}
            \item wolne rozprzestrzenianie się informacji o zmianach w topologii sieci (wolna zbieżność),
            \item stosunkowo częste (co 30s.) przesyłanie dużych porcji informacji w komunikatach RIP, co obciąża sieć.
            \item Wadą RIPv1 jest to, że nie daje możliwości przesyłania masek. RIP w wersji 1 jest protokołem routowania klasowego. Przyjmowana jest maska taka, jaka jest ustawiona na interfejsie, do którego dotarł wektor odległości z tą informacją. Na granicach klas jednak routery RIP wykonują automatyczną sumaryzację, co przy niepoprawnych konfiguracjach może prowadzić do błędów w routowaniu.
        \end{itemize}

        \item RIPv2 przekazuje maski podsieci, można stosować sieci bezklasowe i podsieci o zmiennym rozmiarze. Umożliwia prostą autentykację (przez hasła). Przekazuje adresy następnego skoku w komunikatach. Część wad RIP pozostała: 16 jako metryka oznaczająca nieskończoność, brak alternatywnych tras.
    \end{itemize}

    \subsection{Wiele protokołów routowania. Odległości administracyjne.}
    W środowisku współdziałania wielu protokołów routowania dla tras, w zależności od tego
    jakie jest ich źródło, ustawiana jest tzw. \textbf{odległość administracyjna}. Odległość jest używana tylko \textbf{wewnętrznie przez router}, wybierane są trasy o \textbf{najmniejszej odległości}.\\

    Standardowe odległości dla tras:\\
    \begin{tabular}{|c|c|}
        \hline
        Źródło trasy & Odległość administracyjna\\
        \hline
        Connected interface & 0\\
        \hline
        Static route & 1\\
        \hline
        Summary EIGRP & 5\\
        \hline
        External BGP & 20\\
        \hline
        Internal EIGRP & 90\\
        \hline
        IGRP & 100\\
        \hline
        OSPF & 110\\
        \hline
        IS-IS & 115\\
        \hline
        RIP & 120\\
        \hline
        EGP & 140\\
        \hline
        Internal BGP & 200\\
        \hline
        Unknown & 255\\
        \hline
    \end{tabular}

    Na routerach można również włączyć redystrybucję tras między protokołami routowania
    według różnych zasad, z odpowiednim przeliczaniem metryk.



\end{document}