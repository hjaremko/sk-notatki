\documentclass[../main.tex]{subfiles}


\begin{document}
    \begin{itemize}
        \item \textbf{dłuższy adres} - 128 bitów
        \item \textbf{złożona hierarchia adresów}.
        \item nowa forma nagłówka
        \begin{itemize}
            \item \textbf{podstawowy nagłówek} jest uproszczony - zawiera 40 oktetów
            \item dodatkowe informację przekazywane są za pomocą \textbf{dodatkowych nagłówków}, które są opcjonalne
            \item dzięki temu router szybciej jest w stanie przetworzyć pakiet
            \item takie rozwiązanie daje więcej możliwości rozszerzenia protokołu
        \end{itemize}
        \item \textbf{fragmentacja} jest \textbf{realizowana przez nadwacę}, nie przez routery pośrednie (jak w IPv4)
        \item zawiera rozwinięte mechanizmy \textbf{autokonfiguracji}
        \item zawiera ulepszone mechanizmy obsługi rozgłaszania grupowego (\textbf{multicast}) oraz nowy rodzaj adresowania – adresy pobliskie \textbf{anycast} (przesłanie do jednego – na ogół najbliższego – routera skonfigurowanego do odbierania określonego adresu typu anycast)
        \item włączone są \textbf{mechanizmy bezpieczeństwa} (włączony standard \textbf{IPSec}).
        \item Wsparcie dla \textbf{QoS} (Quality of Service) – mechanizm rezerwacji zasobów na potrzeby komunikujących się programów
    \end{itemize}

    \subsection{Nagłówek IPv6.}
    \begin{tabularx}{\textwidth}{|l|X|}
        \hline
        \textbf{Version} &
        wersja protokółu (6)\\
        \hline
        \textbf{Traffic Class} &
        używana do określenia priorytetu pakietu\\
        \hline

        \textbf{Flow Label} &
        etykietowanie pakietów wymagających takiego samego
        traktowania przez routery\\
        \hline

        \textbf{Payload Length} &
        wielkość danych za nagłówkiem,

        obejmuje też nagłówki dodatkowe\\
        \hline

        \textbf{Next Header} &
        identyfikuje typ następnego nagłówka, np. TCP, UDP

        lub nagłówka dodatkowego IPv6\\
        \hline

        \textbf{Hop Limit} &
        TTL (Czas życia pakietu)\\
        \hline

        \textbf{Source Address} &
        adres źródłowy\\
        \hline

        \textbf{Destination Address} &
        adres docelowy (puste jeśli został

        załączony dodatkowy nagłówek typu Routing)\\
        \hline
    \end{tabularx}


    \subsubsection{Dodatkowe nagłówki.}
    \begin{itemize}
        \item \textbf{typ} nagłówka dodatkowego jest określony w polu \textbf{Next Header} w nagłówku poprzedzającym
        \item nagłówki dodatkowe są sprawdzane i \textbf{przetwarzane przez węzeł} (router, host) o adresie zawartym w polu \textbf{Destination Address} w nagłówku podstawowym
        \item nagłówki dodatkowe muszą być \textbf{sprawdzane i przetwarzane dokładnie w kolejności występowania}
        \item jeśli węzeł musi przetworzyć nagłówek dodatkowy, ale nie może rozpoznać poprawnej liczby w polu Next Header, powinien odrzucić datagram i wysłać komunikat ICMPv6 Parameter Problem
    \end{itemize}

    \subsubsection{Rodzaje nagłówków.}
    \begin{itemize}
        \item \textbf{IPv6 header} - nagłówek podstawowy.
        \item \textbf{Hop-by-hop Options header} - np. w protokole RVSP, w protokole MLD – Multicast Listener Discovery.
        \item \textbf{Destination Options header} - jeśli użyty jest nagłówek dodatkowy Routing. Służy do przekazania opcji, które będą przetwarzane przez węzeł o adresie zawartym w polu Destination Address w nagłówku podstawowym oraz przez węzły o adresach wymienionych w nagłówku dodatkowym typu Routing.
        \item \textbf{Routing header}
        \begin{itemize}
            \item \textbf{typ 0}: odpowiednik routowania źródłowego z IPv4
            \item \textbf{typ 2}: wykorzystywany do obsługi mobilności do przekazania Home Address mobilnego węzła docelowego).
            \item jest używany do określenia \textbf{listy routerów}, przez które \textbf{powinien przejść} pakiet na drodze do miejsca przeznaczenia.
            \item pierwszy węzeł, który przetwarza nagłówek Routing to węzeł (router), którego adres znajduje się w polu Destination Address nagłówka podstawowego. Węzeł ten zmniejsza o jeden liczbę zawartą w polu Segments Left i przepisuje następny adres z nagłówka Routing do pola Destination nagłówka podstawowego. Ostateczne miejsce docelowe datagramu jest określone przez ostatni adres w nagłówku Routing.
            \item w odróżnieniu od IPv4, w IPv6 \textbf{adres docelowy może się zmieniać} na drodze datagramu – jeśli wykorzystywany jest nagłówek dodatkowy Routing
        \end{itemize}
        \item \textbf{Fragment header}.
        \begin{itemize}
            \item Fragmentacja występuje \textbf{tylko w węźle źródłowym}.
            \item Pakiet ma część, która nie podlega fragmentacji – jest to podstawowy nagłówek i wszystkie nagłówki dodatkowe, które muszą być przetwarzane na drodze datagramu.
            \item Część, która podlega fragmentacji składa się z pozostałych nagłówków dodatkowych, nagłówka warstwy wyższej i danych.
            \item Część niepodlegająca fragmentacji jest na początku każdego fragmentu, potem jest nagłówek Fragment, potem część pofragmentowana.
        \end{itemize}
        \item \textbf{Authentication header}.
        \item \textbf{Encapsulating Security Payload header}.
        \item \textbf{Destination Options header} (wykorzystywany dla opcji, które będą przetwarzane tylko przez ostatecznego odbiorcę pakietu, wykorzystywany w mobilnym IP do przekazania źródłowego adresu home).
        \item \textbf{Mobility header} (wykorzystywany do obsługi mobilności, przy zgłaszaniu).
        \item \textbf{Upper-Layer header} (np. TCP, UDP) <- nagłówek dotyczący danych
    \end{itemize}

    \subsection{Adresacja IPv6.}
    \begin{itemize}
        \item \textbf{osiem 16 bitowych sekcji w zapisie szesnastkowym}, oddzielonych dwukropkami,

        np.: \texttt{fe80:0000:0000:0000:0202:b3ff:fe1e:8329}
        \item długi ciąg zer może być \textbf{raz} połączony i przedstawiony jako dwa dwukropki "\texttt{::}"
        \item w środowisku z węzłami IPv4 oraz IPv6 często \textbf{w adresie IPv6 umieszcza się również IPv4}. Są dwa typy takich adresów:
        \begin{itemize}
            \item \textbf{Typ 1: IPv4-Compatible IPv6 Address} - 96 bitów zero + adres IPv4
            \item \textbf{Typ 2: IPv4-Mapped IPv6 Address} (zalecany) -\\
            \texttt{0000..............................0000 ffff} + IPv4 address
        \end{itemize}
        \item Zamiast identyfikatora sieci (jak było w IPv4), w IPv6 używa się tzw. \textbf{prefiksu}. Prefiks to \textbf{pewna liczba bitów} adresu licząc z lewej strony. Prefiks \textbf{określa} zatem \textbf{sieć}. Prefiks podaje się standardowo jako pewną liczbę sekcji adresu.
    \end{itemize}

    \subsubsection{Typy adresów}
    \begin{itemize}
        \item \textbf{Adres niewyspecyfikowany} - \texttt{::/128}
        \item \textbf{Pętla zwrotna} (Loopback) - \texttt{::1/128}
        \item \textbf{Multicast} - \texttt{ff00::/8 (1111 1111)}
        \item \textbf{Link-local unicast} - \texttt{fe80::/10 (1111 1110 10)}
        \item \textbf{Global unicast}
    \end{itemize}


    \subsubsection{Skąd wziąć adres MAC?}
    \begin{itemize}
        \item Unicastowy adres IP na multicastowy (Solicited Node Multicast address) - \texttt{ff02:0:0:0:0:1:ff\_:\_ \_},
        z trzema najmłodszymi bajtami adresu IP
        \item Multicastowy MAC - \texttt{33:33:\_:\_:\_:\_} z czterema najmłodszymi bajtami adresu multicastowego
        \item Ramka skonstruowana z adresem MAC, z pakietem IP z multicastowym IP w środku, z unicastowym w komunikacie ICMP
        \item Switche się znają na tyle, że ramka w miarę dochodzi tylko tam gdzie trzeba
        \item Komputer otrzymujący ramkę sprawdza czy o niego chodzi w komunikacie ICMP, i unicastowo odpowiada
    \end{itemize}

    \subsection{ICMPv6}
    \textbf{Mechanizmy}:
    \begin{itemize}
        \item \textbf{Wykrywanie sąsiada} - jak wyżej
        \item \textbf{Multicast Listener Discovery} (MLD) - jest odpowiednikiem \textbf{Internet Group Management Protocol} (IGMP) w IPv4
        \item \textbf{Wykrywanie ścieżki MTU} (Path MTU Discovery) - nie musi być implementowane w pełni w każdym węźle.
        Host zaczyna wysyłać pakiety o wielkości MTU dla wykorzystywanego łącza.
        Jeśli dostanie komunikat ICMPv6 Packet Too Big, to zmniejsza rozmiar pakietu.
        Może to wystąpić wielokrotnie w czasie komunikacji do miejsca docelowego.
        Po ewentualnym zmniejszeniu rozmiaru pakietu od czasu do czasu host próbuje wysłać
        większy datagram, aby wykryć możliwość przesłania większego datagramu.
        \item \textbf{IPSec w IPv6}
    \end{itemize}

    \textbf{Mechanizmy przejścia}
    \begin{itemize}
        \item \textbf{6to4}
        \begin{itemize}
            \item polega na pakowaniu pakietów IPv6 w pakiety IPv4. Pole protokołu w nagłówku otrzymuje wartość 41. Tak utworzone pakiety wysyłane są do komputera stanowiącego bramę pomiędzy siecią opartą na IPv4 a „prawdziwą” siecią IPv6.
            \item łączenie sieci 6to4 z innymi sieciami 6to4 i innymi IPv6 poprzez sieć IPv4 w okresie przejściowym
        \end{itemize}
        \item \textbf{6over4}
        \item \textbf{ISATAP}
    \end{itemize}


\end{document}
