\documentclass[../main.tex]{subfiles}
\begin{document}

    Adres IP jest przypisywany do karty sieciowej, nie do komputera.

    Są \textbf{trzy typy adresów IPv4}:
    \begin{itemize}
        \item \textbf{Adresy jednostkowe} (unicast) – pojedynczy interfejs sieciowy (komunikacja one-to-one).
        \item \textbf{Adresy rozgłoszeniowe} (broadcast) – wszystkie węzły w tym samym segmencie sieci (one-to-everyone).
        \item \textbf{Adresy grupowe} (multicast) – jeden lub wiele komputerów w jednej lub w różnych segmentach sieci (one-to-many).
    \end{itemize}

    W \textbf{adresie IP} zapisanym binarnie można wyróżnić \textbf{dwie części}:
    \begin{itemize}
        \item \textbf{Identyfikator sieci} (Network ID) - pewna liczba bitów z lewej strony adresu
        \item \textbf{Identyfikator hosta} (Host ID) - pozostałe bity.
    \end{itemize}
    \textbf{Granica} między identyfikatorem sieci a identyfikatorem hosta może być wyznaczona przez
    tzw. \textbf{maskę sieci}.

    Adres IP, który zawiera \textbf{same zera} w części hosta jest traktowany jako \textbf{adres sieci}.
    \textbf{Adresy rozgłoszenia do sieci lub podsieci mają jedynki tylko w części hosta}.

    \textbf{Adres ograniczonego rozgłoszenia} - 255.255.255.255- adres rozgłoszenia
    w danym segmencie sieci ograniczonym routerami.\\

    \subsection{Adresowanie oparte na klasach}

    Pierwszy bajt adresu determinuje do jakiej klasy należy sieć.

\begin{adjustbox}{width=\columnwidth,center}
    \begin{tabular}{|c|c|c|c|c|}
        \hline
        Klasa & Adres sieci & Adresy & Zakres 1-go bajtu & Najstarsze bity\\
        \hline
        A & w.0.0.0 & 1.0.0.0 - 126.0.0.0 & 1 – 126 & 0\\
        \hline
        B & w.x.0.0 & 128.0.0.0 - 191.255.0.0 & 128 – 191 & 10\\
        \hline
        C & w.x.y.0 & 192.0.0.0 - 223.255.255.0 & 192 – 223 & 110\\
        \hline
        D & nie dotyczy & nie dotyczy & 224 – 239 & 1110\\
        \hline
        E & nie dotyczy & nie dotyczy & 240 – 255 & 11110\\
        \hline
    \end{tabular}
\end{adjustbox}

\begin{adjustbox}{width=\columnwidth,center}
    \begin{tabular}{|c|c|c|c|c|c|c|}
        \hline
        Klasa & Ilość sieci & Komp. w sieci & ID sieci & ID hosta & "pierwszy" & "ostatni"\\
        \hline
        A & 126 & $2^{24}-2$ & 1 bajt & 3 bajty & w.0.0.1 & w.255.255.254\\
        \hline
        B & $(191-128+1)*256$ & $2^{16}-2 = 65 534$ & 2 bajty & 2 bajty & w.x.0.1 & w.x.255.254\\
        \hline
        C & $(192-223+1)*256*256$ & $2^8 -2 = 254$ & 3 bajty & 1 bajt & w.x.z.1 & w.x.z.254\\
        \hline
    \end{tabular}
\end{adjustbox}

    \begin{itemize}
        \item \textbf{Adresy klasy D} - przeznaczone są do transmisji grupowych.
        \item \textbf{Adresy klasy E} - zarezerwowane (nie wykorzystywane normalnie do transmisji pakietów).
        \item \textbf{Adresy pętli zwrotnej} (loopback) - postaci 127.x.y.z (na ogół 127.0.0.1). Cały ruch przesyłany na ten adres nie wychodzi z komputera.
    \end{itemize}


    \subsection{Adresowanie bezklasowe}
    Dzielenie na podsieci z \textbf{użyciem dowolnej liczby jedynek}. Do określenia sieci należy podać adres
    sieci oraz maskę. Obecnie w Internecie powszechnie jest wykorzystywane adresowanie
    bezklasowe.

    \begin{itemize}
        \item Protokół \textbf{warstwy trzeciej} modelu ISO OSI.
        \item Oprogramowanie implementujące protokół IP jest odpowiedzialne za:
        \begin{itemize}
            \item \textbf{adresowanie} IP,
            \item \textbf{tworzenie datagramów} IP (pakietów)
            \item uczestniczenie w \textbf{kierowaniu ich} w sieci z punktu początkowego do punktu docelowego.
        \end{itemize}
        \item Realizuje usługę \textbf{zawodną}. Jeśli komunikacja powinna zawierać mechanizmy niezawodności, to muszą one być dostarczone przez protokoły warstwy wyższej.
        \item Datagram IP składa się z nagłówka (header) i bloku danych (payload).
        \begin{itemize}
            \item \textbf{Nagłówek} dzięki informacjom w nim zawartym umożliwia obsługę routowania, identyfikację bloku danych, określenie rozmiaru nagłówka i datagramu oraz obsługę fragmentacji. W nagłówku mogą się znaleźć również tzw. opcje rozszerzające.
            \item \textbf{Blok danych}.
        \end{itemize}
    \end{itemize}

    \subsection{Nagłówek IPv4}
    \begin{itemize}
        \item \textbf{Wersja} (4 bity) (=0100)
        \item \textbf{Długość nagłówka IP} (IHL – Internet Header Length)
        \item \textbf{Typ usługi}: TOS (Type of Service) lub DS. (Differentiaded Services)\\
        Flagi:
        \begin{itemize}
            \item małe opóźnienie
            \item duża przepustowość
            \item niezawodność
        \end{itemize}
        \begin{itemize}
            \item Bits 0-5: DCSP (priorytet + flagi)
            \item Bits 6-7: ECN
        \end{itemize}
        ECN jest rozszerzeniem protokołów IP oraz TCP. Umożliwia powiadamianie punktów końcowych IP/TCP o nadchodzącym zatorze bez usuwania pakietów, poprzez ustawienie warość 11 na bitach ECN. Jest opcjonalny.

        \item \textbf{Długość całkowita} - na podstawie tego pola oraz pola Długość nagłówka można określić wielkość bloku danych oraz początek tego bloku.
        \item \textbf{Identyfikator} kolejnych datagramów. Wartość jest wpisywana przez host nadający i dla kolejnych datagramów jest zwiększana.
        \item \textbf{Flagi} - 3 bity tworzące dwie flagi używane przy fragmentacji datagramów.
        \item \textbf{Przesunięcie fragmentu}
        \item \textbf{Czas życia} (TTL)
        \item \textbf{Protokół} - ICMP, UDP, TCP.
        \item \textbf{Suma kontrolna nagłówka}
        \item \textbf{Adres IP źródła}
        \item \textbf{Adres IP docelowy}
        \item \textbf{Dodatkowe opcje i wypełnienie}
        Opcje mogą zająć maksymalnie 40 bajtów i mogą zawierać m.in.:
        \begin{itemize}
            \item zapis trasy przez którą przeszedł datagram
            \item zapis czasu – timestamp - trasa, czas przejścia.
            \item routowanie źrodłowe\\
            Normalnie to routery wybierają dynamicznie trasę datagramów. Można jednak określić trasę datagramu w opcjach nagłówka IP.
            \begin{itemize}
                \item  dokładne – wysyłający komputer określa dokładną trasę, jaką musi przejść datagram. Jeśli kolejne routery na tej trasie są przedzielone jakimś innym routerem, to wysyła komunikat ICMP „source route failed” i datagram jest odrzucany.
                \item swobodne – wysyłający określa listę adresów IP, przez jakie musi przejść datagram, ale datagram może przechodzić również przez inne routery.
            \end{itemize}
        \end{itemize}
    \end{itemize}
    Za nagłówkiem IP w datagramie znajdują się dane (segment TCP, datagram UDP, komunikat ICMP).

    \subsection{Fragmentacja datagramów IPv4}
    \textbf{MTU} (Maximum Transmission Unit) to największa porcja danych, jaka może być przesłana w ramce przez pewną sieć
    przy wykorzystaniu konkretnej technologii. Jeśli datagram IP jest większy niż wynika to z MTU dla warstwy łącza, to IP dokonuje \textbf{fragmentacji}.
    Najmniejsze MTU po drodze przejścia datagramu nazywa się \textbf{ścieżką MTU}.
    Fragmenty też mogą być dalej dzielone, stają się samodzielnymi pakietami.

    Pole identyfikator w nagłówku IP zawiera numer wysłanego pakietu. Pole powinno być inicjowane przez protokół warstwy wyższej. Warstwa IP zwiększa identyfikator
    o 1 dla kolejnych pakietów.\\

    Jeśli flaga jest ustawiona na Don't Fragment to znaczy, że pakiet nie może być dzielony. W przypadku konieczności dzielenia jest odrzucany i do nadawcy wysyłany jest
    komunikat ICMP (typ 3 z polem kod = 4).

    Jeśli zgubiony zostanie chociaż jeden fragment, wówczas cały wyjściowy pakiet jest odrzucony, więc fragmentacja jest niekorzystna.
    Do tego może ona bardzo obciążać routery.

\end{document}
