\documentclass[../main.tex]{subfiles}
\begin{document}

    Adres IP jest przypisywany do karty sieciowej, nie do komputera.

    Są \textbf{trzy typy adresów IPv4}:
    \begin{itemize}
        \item \textbf{Adresy jednostkowe} (unicast) – pojedynczy interfejs sieciowy (komunikacja one-to-one).
        \item \textbf{Adresy rozgłoszeniowe} (broadcast) – wszystkie węzły w tym samym segmencie sieci (one-to-everyone).
        \item \textbf{Adresy grupowe} (multicast) – jeden lub wiele komputerów w jednej lub w różnych segmentach sieci (one-to-many).
    \end{itemize}

    W \textbf{adresie IP} zapisanym binarnie można wyróżnić \textbf{dwie części}:
    \begin{itemize}
        \item \textbf{Identyfikator sieci} (Network ID) - pewna liczba bitów z lewej strony adresu
        \item \textbf{Identyfikator hosta} (Host ID) - pozostałe bity.
    \end{itemize}
   \textbf{Granica} między identyfikatorem sieci a identyfikatorem hosta może być wyznaczona przez
    tzw. \textbf{maskę sieci}.

    Adres IP, który zawiera \textbf{same zera} w części hosta jest traktowany jako \textbf{adres sieci}.
    \textbf{Adresy rozgłoszenia do sieci lub podsieci mają jedynki tylko w części hosta}.

    \textbf{Adres ograniczonego rozgłoszenia} - 255.255.255.255- adres rozgłoszenia
    w danym segmencie sieci ograniczonym routerami.\\

    \subsection{Adresowanie oparte na klasach}

    Pierwszy bajt adresu determinuje do jakiej klasy należy sieć.

    \begin{tabular}{|c|c|c|c|c|}
        \hline
        Klasa & Adres sieci & Adresy & Zakres 1-go bajtu & Najstarsze bity\\
        \hline
        A & w.0.0.0 & 1.0.0.0 - 126.0.0.0 & 1 – 126 & 0\\
        \hline
        B & w.x.0.0 & 128.0.0.0 - 191.255.0.0 & 128 – 191 & 10\\
        \hline
        C & w.x.y.0 & 192.0.0.0 - 223.255.255.0 & 192 – 223 & 110\\
        \hline
        D & nie dotyczy & nie dotyczy & 224 – 239 & 1110\\
        \hline
        E & nie dotyczy & nie dotyczy & 240 – 255 & 11110\\
        \hline
    \end{tabular}


    \begin{tabular}{|c|c|c|c|c|c|c|}
        \hline
        Klasa & Ilość sieci & Komp. w sieci & ID sieci & ID hosta & "pierwszy" & "ostatni"\\
        \hline
        A & 126 & $2^{24}-2$ & 1 bajt & 3 bajty & w.0.0.1 & w.255.255.254\\
        \hline
        B & $(191-128+1)*256$ & $2^{16}-2 = 65 534$ & 2 bajty & 2 bajty & w.x.0.1 & w.x.255.254\\
        \hline
        C & $(192-223+1)*256*256$ & $2^8 -2 = 254$ & 3 bajty & 1 bajt & w.x.z.1 & w.x.z.254\\
        \hline
    \end{tabular}

    \begin{itemize}
        \item \textbf{Adresy klasy D} - przeznaczone są do transmisji grupowych.
        \item \textbf{Adresy klasy E} - zarezerwowane (nie wykorzystywane normalnie do transmisji pakietów).
        \item \textbf{Adresy pętli zwrotnej} (loopback) - postaci 127.x.y.z (na ogół 127.0.0.1). Cały ruch przesyłany na ten adres nie wychodzi z komputera.
    \end{itemize}


    \subsection{Adresowanie bezklasowe}
    Dzielenie na podsieci z \textbf{użyciem dowolnej liczby jedynek}. Do określenia sieci należy podać adres
    sieci oraz maskę. Obecnie w Internecie powszechnie jest wykorzystywane adresowanie
    bezklasowe.


\end{document}