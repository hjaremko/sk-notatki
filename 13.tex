\documentclass[../main.tex]{subfiles}


\begin{document}

    Metryką jest szybkość łącza.

    Wewnątrz obszaru:
    \begin{itemize}
        \item routery zgłaszają wszystkie połączenia które mają do sąsiadów i tak kaskadowo rozsyłają info o wszystkich
        \item updaty o zmianach
        \item każdy ma swój własny obraz sieci, użycie Dijkstry
    \end{itemize}

    Na brzegach są border routery.
    Wszystkie area połączone do Area 0.

    Rodzaje obszarów:
    \begin{itemize}
        \item “normalne” - bez stuba
        \item Stub Area – do takiego obszaru NIE są wprowadzane trasy zewnętrzne, natomiast sumy tras z innych obszarów są wprowadzane.
        \item Totally Stubby Area – do takiego obszaru nie są wprowadzane ani trasy zewnętrzne, ani sumy tras z innych obszarów OSPF. Wyjście z takiego obszaru jest tylko przez trasę domyślną
        \item Not So Stubby Area (NSSA) – obszar Stub, do którego wprowadzane są pewne (na ogół nieliczne) trasy zewnętrzne, które następnie przekazywane są do innych obszarów tak jak sumy tras.
        \item Not So Stubby Totally Stubby Area – obszar połączenie NSSA i Totally Stabby Area. Routery ABR na granicach różnych obszarów powinny być odpowiednio skonfigurowane, co stanowi dodatkową trudność w konfigurowaniu OSPF.
    \end{itemize}
    Trasy zewnętrzne - z innych protokołów routingu.

    Gateway of last resort - router do którego idziemy kiedy nie mamy trasy.


\end{document}