\documentclass[../main.tex]{subfiles}


\begin{document}
    \textbf{Koncentratory} - huby, dostają info na jeden port i \textbf{wzmacniają go i wysyłają na
    wszystkie inne}. Nie analizuje ramek, fizyczna budowa gwiazdy a logicznie magistrali,
    \textbf{1 warstwa} ISO OSI.

    \textbf{Mosty} - analizują ramki, \textbf{2 warstwa} ISO OSI, ma \textbf{filtrować ruch} tak by niektóre
    ramki pozostawić w jednym segmencie(jeśli MAC jest w tym segmencie), uczy się MAC.

    \textbf{Przełącznik} - analizuje ramki, \textbf{2 warstwa}, ma kilka portów też uczy sie
    MAC-ów (jak zna to tam wysyła a jak nie zna to wysyła wszędzie oprócz tego co dostało).
    \textbf{Tryby działania}:
    \begin{itemize}
        \item \textbf{Store and Forward} - pobiera całą ramkę, sprawdza czy błędna i dopiero potem dalej
        \item \textbf{Cut-through}
        \begin{itemize}
            \item \textbf{Fast Forward Switching} - po otrzymaniu MAC od razu ją nadaje, nie sprawdza błędów, albo czy nie nastąpiła kolizja
            \item \textbf{Fragment-Free Switching} - ramka jest przesyłana dopiero jak dojdzie 64 bajty
        \end{itemize}
    \end{itemize}

    \textbf{Burza broadcastów} - cały czas są robione broadcasty i zapychają sieć(jak mamy
    3 przełączniki w trójkąt połączone i A wyśle broadcast to odbiorą go B i C, nie wyślą go do
    A(bo stamtąd przyszło), ale wyślą do siebie, co z powrotem wyśle do A i tak w kółko).

    \textbf{STP - SpanningTree Protocol} - jak łączymy przełączniki to chcemy
    redundancje między nimi i to jest fajne, ale przez to są pętle (burza broadcastów,
    double packety, złe MAC) i już nie jest, na ratunek STP - będzie redundancja, ale w
    danej chwili tylko jedno ze zduplikowanych jest aktywne.\\
    \textbf{Działanie}:
    \begin{itemize}
        \item \textbf{Wybór korzenia} - na podstawie najniższego priorytetu przełącznika
        \item \textbf{Wybór root-portów} (komunikacja z korzeniem) - jak jest połączony z rootem to jest automatycznie, inne wybierane na podstawie szerokości pasma
        \item \textbf{Wybór portów wyznaczonych} - wszystkie inne połączenia muszą “zdecydować” w którą stronę będą działać i porównują sobie BID i wybierają niższy, a ten drugi port jest robiony na disabled
    \end{itemize}

    \textbf{Stany portów:}
    \begin{itemize}
        \item \textbf{Blocking} - nie przekazuje normalnych, czyta BPDU
        \item \textbf{Listening} - zwykłe ramki nie przekazywane, BPDU czytane i wysyłane
        \item \textbf{Learning} - zwykłe ramki nie przekazywane, BPDU czytane i wysyłane, uczy się MAC
        \item \textbf{Forwarding} - wszystko przekazywane, czytane
        \item \textbf{Disabled} - przez admina
    \end{itemize}

    \textbf{EtherChannel} - połączenie kilku ethernetowych łącz fizycznych w jedno logiczne.
    Przełączniki mogą wówczas równomiernie rozkładać obciążenie na łączu.

    \textbf{Shortest Path Bridging} - rozwinięcie STP, wiele redundantnych ścieżek jest
    wykorzystywanych jednocześnie, link-state do wyznaczenia najkrótszych ścieżek w warstwie
    drugiej.

    \textbf{Przełącznik 3 warstwy} - to urządzenie sieciowe podobne w działaniu do
    routera. Decyzje routingowe podejmowane są na podstawie danych z trzeciej warstwy modelu
    OSI. Do wyznaczania trasy używany jest pierwszy pakiet z danego przepływu a reszta
    pakietów z danego przepływu przełączana jest już w warstwie 2. W związku z tym przełącznik
    warstwy trzeciej nie ma pełnej funkcjonalności routera. Ograniczenia dotyczą m.in.
    translacji adresów (NAT), implementacji mechanizmów bezpieczeństwa czy niestandardowych
    protokołów. Przełącznik warstwy trzeciej jest zazwyczaj droższy i szybszy w działaniu
    niż router. \textbf{Wykonują routowanie między VLANAMI.}

    \textbf{VLAN} - wirtualne sieci LAN tworzone są przy pomocy przełączników. Są
    konfigurowane programowo, zatem ewentualne zmiany konfiguracji nie wymagają zmian w
    okablowaniu. Komputery z każdej utworzonej sieci wirtualnej nie muszą być dołączone
    bezpośrednio do jednego przełącznika. Urządzenia należące do jednej sieci wirtualnej
    mogą się ze sobą komunikować tak jakby były w jednym segmencie sieci LAN, działa np. ARP.

    \textbf{Trunk} - \textbf{połączenia między przełącznikami}.
    Połączenie typu trunk umożliwia przekazywanie ramek należących do różnych sieci VLAN
    po jednym fizycznym nośniku. Za kontrolę połączenia trunk odpowiada \textbf{protokół VTP}
    (VLAN Trunking Protocol). Łączenie danych w jeden wspólny kanał, w którym przesyłane
    są dane.

    \textbf{VTP - Virtual Trunking Protocol} - umożliwia zautomatyzowaną konfigurację
    wielu przełączników z jednego miejsca na drodze wymiany odpowiednich ramek z
    sąsiadującymi przełącznikami.
\end{document}