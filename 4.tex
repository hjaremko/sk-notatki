\documentclass[../main.tex]{subfiles}


\begin{document}

    \begin{table}[H]
        \begin{center}
            \begin{tabular}{ | p{.2\textwidth} | p{.1\textwidth} | p{.1\textwidth} | p{.15\textwidth} | p{.2\textwidth} | p{.2\textwidth} | }
                \hline
                Preambuła & src MAC & dst MAC & Typ danych & 46-1500 Dane & CRC\\
                \hline
                żeby się karty

                sieciowe

                zsynchronizowały & & & & tu jest pakiet IP i

                segment TCP & suma kontrolna\\
                \hline
                warstwa 1 & \multicolumn{5}{c}{warstwa 2 }|\\
                \hline
            \end{tabular}
        \end{center}
    \end{table}

    sposób dostępu do nośnika - kiedyś przy topologii magistrali był half-duplex (albo tylko nadajesz, albo tylko odbierasz, tylko jeden komputer może nadawać żeby się nie pokrywały)

    CSMA - Carrier Sense Multiple Access - komputer nasłuchuje czy ktoś inny nadaje, jak nie to dopiero sam zaczyna nadawać; żąda potwierdzenia otrzymania ramek; jak ktoś nadaje to czeka pseudolosowy czas i próbuje znowu

    teraz już nie ma problemu z kolizjami bo mamy switche
    kabel cross - jak łączymy bez switcha dwa urządzenia
    kabel prosty - jak jest switch

    teraz sobie urządzenia wykrywają czy są dobrze połączone i same korygują żeby było ok


    CSMA CD (Colission Detect) - nasłuchują  nadając, żeby wiedzieć czy jest kolizja; jak jest kolizja to wysyłają sygnał zagłuszający, żeby powiedzieć że była kolizja i dane wysłane przed chwilą to śmieci




\end{document}